%---------------FICHIER DE CONFIGURATION DU TRAVAIL DE MATURITÉ--------------------------
% Vous trouverez les instructions de compilation du fichier source principal main.tex
% dans la documentation se trouvant dans le fichier DocumentationTMlatex.pdf

% Il est indispensable de remplir ce fichier !!!!!!!!!

%----------------Version provisoire---------------------------------------------------------

% Par défaut le modèle est en version provisoire. Un filigrane est placé
% en haut à gauche de chaque page, les notes de todo en marges sont activées,
% un rectangle noir se trouve à la fin des lignes trop longues, ...
%%%%%%%%%%%%%%%%%%%%%%%%%%%%%%%%%%%%%%%%%%%%%%%%%%%%%%%%%%%%%%%%%%%%%%%%%%%%%%%%%%%%%%%%%%%
% Pour la version finale n'oubliez surtout pas de retirer la mention "draft"
% de la déclaration de documentclass au tout début du fichier main.tex.
%%%%%%%%%%%%%%%%%%%%%%%%%%%%%%%%%%%%%%%%%%%%%%%%%%%%%%%%%%%%%%%%%%%%%%%%%%%%%%%%%%%%%%%%%%%

%---------------La langue----------------------------------------------------------------
% Le choix de la langue est très important, car les différents titres de sections vont
% être traduits automatiquement. Cependant, par la suite vous pourrez toujours en donner
% vous même une traduction et l'activer à la place de celle par défaut.

\newcommand{\lalangue}{francais}
%\newcommand{\lalangue}{english}
%\newcommand{\lalangue}{deutch}

%---------------La page de titre---------------------------------------------------------

% Le titre du travail de maturité.
\newcommand{\worktitle}{Les Virus}
% Le sous-titre du travail. S'il n'y a pas de soustitre mettez :
%\newcommand{\worksubtitle}{}
% Pour un sous-titre sur plusieurs lignes, mettez un double back slash
\newcommand{\worksubtitle}{Expérimentations \\
                           sur plusieurs langages informatiques}

% Texte au milieu de la page de titre.
\newcommand{\worktype}{Travail de maturité}

% L'auteur.
\newcommand{\theauthor}{Simoes da Silva Flavio}

% La date de soumission du travail.
\newcommand{\workdateyear}{2019}
\newcommand{\workdatemonth}{23 janvier}

% L'indication pour le mentor
\newcommand{\supervisorslabel}{Mentor}
% le prénom et le nom du mentor en petites capitales
\newcommand{\worksupervisors}{
    Vincent \textsc{Guyot}
}

%----------------Le pied de page-----------------------------------------------------------

% Nom de l'école (au choix, vous pouvez aussi en mettre d'autres)
\newcommand{\ecole}{Lycée Blaise-Cendrars}

% Adresse sur une seule ligne (idem)
%\newcommand{\adressedelecole}{Adresse de votre école}

% Décommentez si vous voulez que les deux images de logos au bas de la page de titre ne soient pas présentes.
% Par défaut elles le sont.
%\newcommand{\leslogos}{N}

% Choisissez si vous faites votre travail en option spécifique, en option complémentaire ou dans les ateliers interdisciplinaires

%\newcommand{\loption}{Option spécifique}
%\newcommand{\loption}{Specific option}
%\newcommand{\loption}{Spezifische Option}
\newcommand{\loption}{Option complémentaire}
%\newcommand{\loption}{Additional option}
%\newcommand{\loption}{Zusätzliche Möglichkeit}
%\newcommand{\loption}{Atelier interdisciplinaire}
%\newcommand{\loption}{Interdisciplinary workshop}
%\newcommand{\loption}{Interdisziplinärer Arbeitsgruppe}

% Choisissez votre domaine en décommentant la bonne ligne ou en en rajoutant une.

%newcommand{\domaine}{Physique}
%\newcommand{\domaine}{Physics}
%\newcommand{\domaine}{Physik}

\newcommand{\domaine}{Informatique}
%\newcommand{\domaine}{Informatics}
%\newcommand{\domaine}{Informatik}

%\newcommand{\domaine}{Mathématiques}
%\newcommand{\domaine}{Mathematics}
%\newcommand{\domaine}{Mathematik}

%---------------Citations, remerciements, acronymes, licence, site officiel, cdrom, index------------

% Par défaut, il y a une page pouvant contenir des pensées. Décommentez si vous n'en voulez pas.
% Le fichier à remplir du texte désiré est pagesspeciales/citations.tex
%\newcommand{\unepensee}{N}

% Par défaut, il y a une page pouvant de dédicaces. Décommentez si vous n'en voulez pas.
% Le fichier à remplir du texte désiré est pagesspeciales/remerciements.tex
%\newcommand{\unededicace}{N}

% Par défaut, il y a une page de résumé. Décommentez si vous n'en voulez pas.
% Le fichier à remplir du texte désiré est pagesspeciales/resume.tex
%\newcommand{\unresume}{N}

% Par défaut, il y a une page d'acronymes. Décommentez si vous n'en voulez pas.
% Les acronymes sont à entrer dans le texte par \ac{BBC}
% et le fichier à remplir pour en donner la traduction est pagesspeciales/acronymes.tex
%\newcommand{\unacronyme}{N}

% Par défaut, il y a une page de licence. Décommentez si vous n'en voulez pas.
% Le fichier à remplir pour décrire le site est pagesspeciales/licence.tex
%\newcommand{\unelicence}{N}

% Par défaut, il y a une page sur le site du travail. Décommentez si vous n'en voulez pas.
% Le fichier à remplir pour le texte de la licence est pagesspeciales/website.tex
%\newcommand{\unsite}{N}

% Par défaut, il y a une page sur le cdrom du travail. Décommentez si vous n'en voulez pas.
% Le fichier à remplir pour le texte de la licence est pagesspeciales/cdrom.tex
%\newcommand{\uncdrom}{N}

% Par défaut, il y a une page d'index. Décommentez si vous n'en voulez pas.
% Il faut alors mettre les clés d'index avec \index{mot} et faire un make index.
\newcommand{\unindex}{N}

% Par défaut, il y a une page d'annexes. Décommentez si vous n'en voulez pas.
% Vous pouvez utiliser la page d'annexes contenue dans le répertoire "chapitre"
% pour y mettre vos annexes.
%\newcommand{\lesannexes}{N}

% Par défaut, il y a une page de notes de fin. Décommentez si vous n'en voulez pas.
\newcommand{\lesnotesdefin}{N}

%----------------Chapitres------------------------------------------------------------------------------
% Le nombre de chapitres désirés
\newcommand{\nbchap}{3}

%----------------Annexes------------------------------------------------------------------------------
% Le nombre d'annexes désirées
\newcommand{\nbannexes}{3}

%----------------Listes des figures, tables et listings-------------------------------------------------
% Définition du titre de la liste des figures.

% Si rien n'est défini, c'est-à-dire si la commande est commentée, c'est le module de gestion des
% langues babel qui choisit la traduction, sauf pour le français où le choix s'est porté sur :
% "Liste des figures" au lieu de "Table des figures" proposé par babel.
% Si la traduction ne vous convient pas, vous pouvez mettre ce
% que vous voulez pour la liste des figures. La commande doit alors être décommentée.
%\newcommand{\titrelistedesfigures}{Liste des figures} % Car nom babel = Table des figures

% Définition du titre de la liste des tableaux.
% Si rien n'est défini, c'est-à-dire si la commande est commentée, c'est le module de gestion des
% langues babel qui choisit la traduction.
% Mais si la traduction ne vous convient pas, vous pouvez mettre ce que vous voulez pour la liste
% des tableaux. La commande doit alors être décommentée.
%\newcommand{\titrelistedestables}{Liste des tables}

% Définition du titre de la liste des codes source.
% Si rien n'est défini, c'est-à-dire si la commande est commentée, les traductions sont :
% Liste des codes sources, Listings, Liste der Quellcodes.
% Mais si la traduction ne vous convient pas, vous pouvez mettre ce que vous voulez pour la liste
% des codes sources. La commande doit alors être décommentée.
%\newcommand{\titrelistedescodes}{Liste des codes sources}

% Définition du titre des notes finales.
% Si rien n'est défini, c'est-à-dire si la commande est commentée, les traductions sont :
% Notes finales, Endnotes, Endenoten.
% Mais si la traduction ne vous convient pas, vous pouvez mettre ce que vous voulez pour la liste
% des codes sources. La commande doit alors être décommentée.
%\newcommand{\titrelistedesnotesdefin}{\textnormal{Liste des notes}}

% Par défaut, il y a une liste des figures. Décommentez si vous n'en voulez pas.
%\newcommand{\unelistefig}{N}

% Par défaut, il y a une liste des crédits. Décommentez si vous n'en voulez pas.
%\newcommand{\unelistecredits}{N}

% Par défaut, il y a une liste des tables. Décommentez si vous n'en voulez pas.
\newcommand{\unelistetbl}{N}

% Par défaut, il y a une liste des codes sources. Décommentez si vous n'en voulez pas.
%\newcommand{\unelistelst}{N}

%----------------Flottants-------------------------------------------------------------
% Définitions des préfixes des préfixes des références aux figures, tables et listings.
% Par exemple : ... voir FIGURE 3.1 où FIGURE 3.1 est mis automatiquement et traduit en
% fonction de la langue.
% English: Figure, Table, Listing (default)
% German: Abbildung, Tabelle, Listing
% French: Figure, Table, Listing
% Aucun préfixe n'est mis pour permettre des références du type :
% ... voir figures 3.1, 3.2 et 3.3. Ici seuls les numéros sont des références.
% Pour changer le nom des titre des références des légendes décommentez à souhait
%\addto\captionsfrench{\def\figurename{Graphique}}
%\addto\captionsfrench{\def\tablename{Tableau}}

%----------------Figures----------------------------------------------------------------

% Chemin vers le répertoire des figures. Vous pouvez ajouter d'autres chemins entre accolades
% à l'intérieur des accolades principales.
\graphicspath{{./images/}}

% Insertion de figures ; copier coller dans les chapitres en fonction des besoins
%\tmfigureB{NomFigureSansExtension}{Légende}{fig:votreLabel}	% Taille grande
%\tmfigureN{NomFigureSansExtension}{Légende}{fig:votreLabel}	% Taille normale
%\tmfigureS{NomFigureSansExtension}{Légende}{fig:votreLabel}	% Taille petite
%\tmfigureT{NomFigureSansExtension}{Légende}{fig:votreLabel}	% Taille très petite

%----------------Insertion de code : listings-----------------------------------------------

% Seul latex est appelé par défaut. On peut charger d'autres languages pour les listings
% selon le modèle de la ligne suivante. Voir le package listings pour les langages disponibles.
%\lstloadlanguages{HTML,PHP,TeX}
% Il faut ensuite placer [...,language=HTML,...] après l'appel \begin{lstlisting} pour définir
% le language à utiliser dans un script particulier
% Aussi on peut mettre du code latex pour référencer le numéro des lignes. Il faut alors avoir
% recours à un caractère d'échappement sur le modèle suivant :
% \lstloadlanguages{Python}
% \lstset{language=Python,escapechar=|}
% Après le caractère d'échappement, on peut alors placer une commande \label{} de référencement
% qui permet d'avoir accès au numéro de la ligne dans le texte via \ref{}.
\lstloadlanguages{TeX}

%----------------Le texte-------------------------------------------------------------------

% Réglage de l'indentation de la première ligne de chaque paragraphe
%\parindent=0in		% pas d'indentation
\parindent=0.1in	% petite indentation
%\parindent=0.2in	% moyenne indentation
%\parindent=0.3in	% grande indentation

% Si vous désirez commenter du code latex sur plusieurs lignes, utilisez la forme :
%\begin{comment} ... \end{comment}

%-----------------Un index------------------------------------------------------------------

% Il n'est pas nécessaire de réaliser un index. Par défaut, il n'y en a pas. Décommentez si vous en voulez un.
% Pour construire le fichier d'index idx, il doit être compilé avec makeindex en ligne de commande.
%\usepackage{makeidx} \makeindex
