% Attention, ce chapitre sera le chapitre 5,
% contrairement à ce que son nom de fichier pourrait faire penser !

\chapter{Conclusion}

\section{Évolution}
Après deux semaines de travail sur ce sujet, je n'imaginais pas pouvoir faire des expérimentations. Je ne me sentais pas réellement capable de le faire et encore moins de les comprendre. Mais avec un peu de temps et de volonté, j'ai commencé à apprécier de créer certains virus parce qu'il existe d'innombrables méthodes et différentes manières de s'amuser avec les scripts. Il y a également quelque chose qui, plus le travail avançait, plus m'a donné de plaisir: l'apprentissage de nouveaux langages qui donnent une diversité encore plus grande. Non pas qu'un seul langage ne soit pas suffisant, mais on récolte de plus en plus d'informations et on s'habitue à tout type d'écriture. Et ceci va de même avec les changement de système d'exploitation: j'estime qu'il faut, maintenant, être capable de changer d'os et de connaître un maximum dans chacuns d'eux. Mes objectifs personnels ont été remplis malgré que la connexion avec ssh ne se soit pas passée comme souhaité, ça n'a pas toujours été facile mais ceci m'a apporté de nouvelles connaissances. 
\section{Conclusion}

La partie théorique m'a aidé à comprendre ce qu'est réellement un virus. je n'imaginais pas qu'un simple malware pourrait engendré d'énormes coûts à des entreprises ou même à des multinationales. Maintenant, la partie où se trouvent les expérimentations démontre qu'il n'est pas si aisé de construire un virus aussi puissant et destructeur que ceux vus précédemment. Pour ma part, je reste convaincu que ces différents essais sur ces multiples langages m'ont appris à comprendre la complexité d'un virus informatique.
 
Il m'a paru plus simple de créer sous Windows; j'y ai notamment trouvé bien plus d'aides sur des forums et beaucoup d'explications sur les défauts -des versions antérieures principalement- de l'os. Il se pourrait également qu'il existe moins de virus sous Linux, Raspian, Mac, etc. dû au fait qu'il y ait moins d'utilisateurs qui employent ces systèmes d'exploitation. Ce qui ne m'a pas facilité la tâche pour ce qui est de tenter de trouver des failles sur ces différents.

Ce travail avait également pour but de montrer que l'on n'est jamais en pleine sécurité. Il ne suffit que d'un click pour télécharger un mauvais dossier qui peut infecter notre ordinateur. Il faut également faire attention à nos mots de passe: éviter de laisser celui par défaut ou en avoir un qui est devinable car un simple keylogger pourrait s'emparer de nos données personnelles. D'autre part, il est vrai qu'une infection peut se produire suite au "plug-in" d'un périphèrique externe, mais celles-ci sont devenues très rares maintenant que les os désactivent l'autorun et limitent les accès lorsqu'on n'est pas administrateur. 